\documentclass[12pt,letterpaper]{article}

%Packages
\usepackage{amsmath}
\usepackage{amsthm}
\usepackage{amsfonts}
\usepackage{amssymb}
\usepackage{amscd}
\usepackage{dsfont}
\usepackage{enumerate}
\usepackage{fancyhdr}
\usepackage{mathrsfs}
\usepackage{bbm}
\usepackage{framed}
\usepackage{mdframed}
\usepackage{cancel}
\usepackage{float}
\usepackage{mathtools}
%\usepackage[]{mcode}
\usepackage{graphicx}
\usepackage{tikz}

%Page formatting
\usepackage[letterpaper,voffset=-.5in,bmargin=3cm,footskip=1cm]{geometry}
\setlength{\parindent}{0.0in}
\setlength{\parskip}{0.1in}
\allowdisplaybreaks
\headheight 15pt
\headsep 10pt

% Common Commands
\newcommand\N{\mathbb N}
\newcommand\Z{\mathbb Z}
\newcommand\R{\mathbb R}
\newcommand\Q{\mathbb Q}
\newcommand\lcm{\operatorname{lcm}}
\newcommand\setbuilder[2]{\ensuremath{\left\{#1\;\middle|\;#2\right\}}}
\newcommand\E{\operatorname{E}}
\newcommand\V{\operatorname{V}}
\newcommand\Pow{\ensuremath{\operatorname{\mathcal{P}}}}

\DeclarePairedDelimiter\ceil{\lceil}{\rceil}
\DeclarePairedDelimiter\floor{\lfloor}{\rfloor}

% 22 style
\newcommand\hint[1]{\textbf{Hint}: #1}
\newcommand\note[1]{\textbf{Note}: #1}
\newenvironment{22enumerate}{\begin{enumerate}[a.]\itemsep0em}{\end{enumerate}}
\newenvironment{22itemize}{\begin{itemize}\itemsep0em}{\end{itemize}}
\fancypagestyle{firstpagestyle} {
  \renewcommand{\headrulewidth}{0pt}%
  \lhead{\textbf{CSCI 0220}}%
  \chead{\textbf{Discrete Structures and Probability}}%
  \rhead{Klivans}%
}

\pagestyle{fancyplain}

\newcommand\EX{\mathds{E}}
\newcommand\PR{\mathds{P}}
\newcommand\zlam{Z_\lambda}
\DeclareMathOperator*{\argmin}{arg\,min}

%%%%%%%%%%%%%%%% COMPILE WITH \soltrue or \solfalse %%%%%%%%%%%%%%%%%%%%%%%%%%%%%
\newif\ifsol
\soltrue  % SOLUTIONS
\solfalse % NO SOLUTIONS
%%%%%%%%%%%%%%%%%%%%%%%%%%%%%%%%%%%%%%%%%%%%%%%%%%%%%%%%%%%%%%

%%%%%%%%%%%%%%%% solution help %%%%%%%%%%%%%%%%%%%%%
\newcommand{\solu}[2]{ \begin{mdframed} \ifsol #2 \else \vspace{#1} \fi \end{mdframed} }
\newcommand{\solt}{ \ifsol (T) \fi }
\newcommand{\solf}{ \ifsol (F) \fi }
\newcommand{\solm}[1]{\ifsol \textit{(#1)} \fi}

\begin{document}
  \thispagestyle{firstpagestyle}
  \begin{center}
    {\large \textbf{Recitation 4}}
    
    {\large Induction and Algorithms}
  \end{center}
  
 \section*{Review}

      We will review the template for an inductive proof.

      For example, say we are trying to prove that $\sum_{i=0}^n i = \frac{n(n+1)}{2}$ is true for all $n \in \N$. 

      \begin{enumerate}
        \item Define the predicate $P(n)$.

        \textit{Let $P(n)$ be the predicate that $\sum_{i=0}^n i = \frac{n(n+1)}{2}$.}

        \item Show that the base case is true.

        \textit{We will first show $P(0)$ is true. $\sum_{i=0}^0 i = 0$ and $\frac{0(0+1)}{2} = 0$ so they are equal as needed.}

        \item Assume the inductive hypothesis is true. If you are using stardard induction then you will assume 
			$P(k)$ is true for some integer $k$. If you are using strong induction then you will assume $P(i)$ is true for all $i \leq k$.

        \textit{Assume $P(k)$ is true for some integer $k$.}

        \item Show that $P(k+1)$ is true given the inductive hypothesis.

		\textit{We will now show that  $\sum_{i=0}^{k+1} i = \frac{(k+1)(k+2)}{2}$}.

       	\textit{We know that $\sum_{i=0}^{k+1} i = \left( \sum_{i=0}^{k} i \right)+ (k+1)$.}

		\textit{By our inductive hypothesis $\sum_{i=0}^k i = \frac{k(k+1)}{2}$.}

         \textit{Therefore}
		\begin{align*}
			\sum_{i=0}^{k+1} i &= \left( \sum_{i=0}^{k} i \right)+ (k+1) \\
			&=  \frac{k(k+1)}{2} + (k+1) \\
			&= \frac{k(k+1) + 2(k+1)}{2} \\
			&= \frac{(k+1)(k+2)}{2}
		\end{align*}
		\textit{as needed.}\qed

	\subsection*{Warm-up}

      \begin{22enumerate}

      \item
      Prove by induction that for all positive integers $n$, there exists a 
      positive integer $m$ such that:
      \[m^2 \leq n < (m + 1)^2\]

	\hint{The inductive step may contain two cases.}

	\solu{17cm}{ Let $P(k)$ denote the property that there exists a positive integer 
        $m$ such that $m^2 \leq k < (m + 1)^2$. 

        Construct a proof by induction.

        \textbf{Base Case:} Consider $k = 1$.

        Let $m = 1$. Then $m^2 = 1$ and $(m + 1)^2 = 4$, so 
        $m^2 \leq k <(m + 1)^2$. Thus $P(1)$ holds.

        \textbf{Inductive Hypothesis:}

        Assume $P(k)$ is true for some positive integer $k$.

        \textbf{Inductive Step:} Consider $n = k+1$. 

        By the inductive hypothesis, there exists a positive integer $m$ 
        such that $m^2 \leq k < (m + 1)^2$. 

        There are two cases:
        \begin{22enumerate}

        \item \textbf{Case 1:} $k + 1 < (m + 1)^2$. Then we can define 
        $m_* = m$, and it holds that $m_*^2 \leq k + 1 < (m_* + 1)^2$. 

        \item \textbf{Case 2:} $k + 1 = (m + 1)^2$. Then we can define 
        $m_* = m + 1$. Then $k + 1 = (m + 1)^2 < ((m + 1) + 1)^2$, so 
        $k + 1 < (m_* + 1)^2$.
        Since $k + 1 = (m + 1)^2$, $k + 1 = m_*^2$, so it is true that 
        $m_*^2 \leq k + 1 < (m_* + 1)^2$.

        \end{22enumerate}

        Because $k < (m + 1)^2$ and $k$ and $m$ are integers, then 
        $k + 1 \ngtr (m + 1)^2$, so these are the only two cases.

        In either case, we found an integer $m_*$ such that 
        $m_*^2 \leq k + 1 < (m_* + 1)^2$. Thus $P(k + 1)$ is correct.

        Thus $P(k)$ implies $P(k + 1)$.

        Since $P(1)$ is true, and since for all positive integers $k$, 
        $P(k)$ implies $P(k+1)$, then by standard induction, $P(n)$ is true for 
        all positive integers $n$.

        Thus, for all positive integers $n$, there exists a positive integer 
        $m$ such that $m^2 \leq n < (m + 1)^2$.}
	
		\pagebreak

      \item
      Prove by contradiction that there exists a \textbf{unique} such $m$.

	 \solu{17cm}{        We prove by contradiction. Assume that for some positive integer 
        $n$, there exist two unique positive integer solutions 
        $m_1 \neq m_2$ such that $m^2 \leq n < (m + 1)^2$. Without loss 
        of generality, assume $m_2 > m_1$. Then $m_2 \geq m_1 + 1$. 
        Because both sides of the equation are non-negative, we can square 
        both sides of this equation to get $m_2^2 \geq (m_1 + 1)^2$.

        By our assumption, we also know that $m_2^2 \leq n$ and that 
        $n < (m_1 + 1)^2$. $m_2^2 \leq n < (m_1 + 1)^2$, so 
        $m_2^2 < (m_1 + 1)^2$. 

        But this is a contradiction, because $m_2^2 \geq (m_1 + 1)^2$. Thus 
        our assumption was false.

        Thus, for all positive integers $n$, there can exist only one unique 
        integer $m$ such that $m^2 \leq n < (m + 1)^2$. 

        From part a, we know that a solution exists for all positive 
        integers $n$. Thus, for all positive integers $n$, there exists 
        exactly one unique integer $m$ such that $m^2 \leq n < (m + 1)^2$.}

      \end{22enumerate}

	\textbf{Checkpoint - Call a  TA over}

    \end{enumerate}

	\section*{Section Lesson - Algorithms}

	\subsection*{Why does induction work?}

	You can think of induction as a ladder. We want to prove that we can reach every step of the ladder.

	The base case says that we can reach the first step of the ladder.

	The inductive hypothesis says that we can get to the $k^{\text{th}}$ step of the ladder.

	The inductive step says that if we can get to the $k^{\text{th}}$ step of the ladder, then we can get to step $k+1$. 

	Therefore, once we get to step 1, we can get to step 2. Once we get to step 2, we can get to step 3. And so on for all steps of the ladder.

	\subsection*{Induction for algorithms}
	
	Induction can also be used to prove the correctness of algorithms. An algorithm is simply a series of steps. 

	Consider the following $SORT$ algorithm which operates on a list of length $n$.
	\begin{enumerate}
		\item If $n=1$ do nothing as the list is already sorted.
		\item Find the smallest element in the list and remove it.
		\item Run the $SORT$ algorithm on the remaining $n-1$ elements.
		\item Put the smallest element at the begining of the list. 
	\end{enumerate}

	We can now prove that $SORT$ correctly sorts a list of size $n$ for $n \geq 1$. 
	

        \textit{Base case: $SORT$ does nothing on a list of size 1, which is already sorted.} 

        \textit{Inductive hypothesis: Assume $SORT$ correctly sorts a list of size $k$.}

	   \textit{Inductive step: We will now show that $SORT$ correctly sorts a list of size $k+1$. In step 3, we are left with a list of size $k$. By our inductive hypothesis, $SORT$ correctly sorts this sublist. Therefore, when we put the smallest element (which we have removed) at the begining, our list is completely sorted as needed.}\qed

Your turn.

\textbf{Given a stack of pancakes, you can stick your spatula in between any two
pancakes, and flip the stack of pancakes above the spatula upside-down. Find an algorithm
to sort $n$ pancakes from largest to smallest, using at most $2n$ flips. Prove it is correct.}

\solu{18cm}{The algorithm is to always find the largest pancake, flip it to the top, then flip the whole stack so the largest pancake is on the bottom. After this proceed recursively on all but the largest pancake. The proof for why this is correct is identical to the proof for $SORT$ given above.}

\textbf{Checkpoint - Call a TA over}

\pagebreak

\textbf{Consider the following inductive proof}



Claim: All $n$ people in the room have the same eye color.
\begin{proof} 

Base case: When $n=1$ clearly everyone in the room has the same eye color as there is only one person in the room.

Inductive Hypothesis: Assume that in a room of $k$ people, everyone in the room has the same eye color.

Inductive Step: We will now show that in a room of $k+1$ people everyone has the same eye color.

We know from our inductive step that the first $k$ people in the room have the same eye color, and that the last $k$ people in the room have the same eye color. Since there is some overlap between the first $k$ people and the last $k$ people, everyone in the room must have the same eye color.
\end{proof}

\textbf{Can you find the flaw in this proof?}

\solu{10cm}{The inductive step does not work for 2 people.}

\pagebreak

\textbf{Challlenge 1:}
100 dragons are sitting in a circle so that every dragon can see every other dragon.

Every dragon has green eyes, but the dragons don't know this. No dragon knows its own eye color, and the dragons cannot talk.

Ben comes and tells the circle of dragons that at least one of them has green eyes. 

If a dragon knows it has green eyes, it will turn into a human at midnight. 

Prove that on night 100, all dragons will turn into humans at once. 

\hint{Use induction! What is the base case? Ignore the number 100 and instead consider $n$ dragons and show they turn into humans on the $n^{\text{th}}$ night.}


\solm{If there is only one dragon, it takes her one night to turn into a human. Now assume it takes $k$ nights for $k$ dragons. Consider the case where there is $k+1$ dragons and consider what one individual dragon is thinking. This one dragon sees $k$ dragons with green eyes. Therefore, if this dragon does not have green eyes, she would expect the other dragons to turn into humans after $k$ night since they can see her. Since they don't, this dragon realizes that she also has green and turns into a dragon on night $k+1$.}

\vspace{6cm} 
\textbf{Challenge 2:}
We say that an infinite set $S$ is countable if there exists a bijection from $\N$ to $S$. Giventhat $\N \times \N$ is countable, prove by induction that $\N^k$ is countable where $k \in \N$.

\solm{$\N^k$ is countable so there exists a bijection $f_k$. We can now make a bijection $g_{k+1}$ from $\N^{k+1}$ to $\N \times \N$. $g_{k+1} ((x_1,...,x_{k+1})) = (f_k(x_1,...,x_k),x_{k+1})$.}
  

\end{document}