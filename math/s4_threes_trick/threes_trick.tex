\documentclass[12pt,letterpaper]{article}

%Packages
\usepackage{amsmath}
\usepackage{amsthm}
\usepackage{amsfonts}
\usepackage{amssymb}
\usepackage{amscd}
\usepackage{dsfont}
\usepackage{enumerate}
\usepackage{fancyhdr}
\usepackage{mathrsfs}
\usepackage{bbm}
\usepackage{framed}
\usepackage{mdframed}
\usepackage{cancel}
\usepackage{float}
\usepackage{mathtools}
%\usepackage[]{mcode}
\usepackage{graphicx}
\usepackage{tikz}
\usetikzlibrary{automata,arrows}

%Page formatting
\usepackage[letterpaper,voffset=-.5in,bmargin=3cm,footskip=1cm]{geometry}
\setlength{\parindent}{0.0in}
\setlength{\parskip}{0.1in}
\allowdisplaybreaks
\headheight 15pt
\headsep 10pt

% Common Commands
\newcommand\N{\mathbb N}
\newcommand\Z{\mathbb Z}
\newcommand\R{\mathbb R}
\newcommand\Q{\mathbb Q}
\newcommand\lcm{\operatorname{lcm}}
\newcommand\setbuilder[2]{\ensuremath{\left\{#1\;\middle|\;#2\right\}}}
\newcommand\E{\operatorname{E}}
\newcommand\V{\operatorname{V}}
\newcommand\Pow{\ensuremath{\operatorname{\mathcal{P}}}}

\DeclarePairedDelimiter\ceil{\lceil}{\rceil}
\DeclarePairedDelimiter\floor{\lfloor}{\rfloor}

% 22 style
\newcommand\hint[1]{\textbf{Hint}: #1}
\newcommand\note[1]{\textbf{Note}: #1}
\newenvironment{22enumerate}{\begin{enumerate}[a.]\itemsep0em}{\end{enumerate}}
\newenvironment{22itemize}{\begin{itemize}\itemsep0em}{\end{itemize}}
\fancypagestyle{firstpagestyle} {
  \renewcommand{\headrulewidth}{0pt}%
  \lhead{\textbf{CSCI 0220}}%
  \chead{\textbf{Discrete Structures and Probability}}%
  \rhead{Klivans}%
}

\pagestyle{fancyplain}

\newcommand\EX{\mathds{E}}
\newcommand\PR{\mathds{P}}
\newcommand\zlam{Z_\lambda}
\DeclareMathOperator*{\argmin}{arg\,min}

\begin{document}
  \thispagestyle{firstpagestyle}
  \begin{center}
    {\large \textbf{Recitation 5}}
    
    {\large Three's Trick}
  \end{center}
  
   \section*{Review}	
	
	\textbf{Defn 1:} We say that $a \mid b$ ($a$ divides $b$) when $b = ka$ for some $k \in \Z$. 

	\textbf{Defn 2:} $a \equiv b \pmod m$ if $m \mid (b-a)$. In other words, $a$ and $b$ have the same remainder upon division by $m$. Convince yourself that these statements are equivelant. 

\textbf{Properties of Congruence Relations:}

For $a,b \in \mathbb{Z}^+$, if $a \equiv b \mod{m}$, then
\begin{itemize}
    \item $a + c\equiv b + c\mod{m}$ for $c \in \mathbb{Z}$
    \item $ac \equiv bc \pmod{m}$ for $c \in \mathbb{Z}$
    \item $a^n \equiv b^n \mod{m}$ for $n \in \mathbb{Z^+}$
\end{itemize}
If we also have $c \equiv d \pmod{m}$, then
\begin{itemize}
    \item $a + c\equiv b + d\mod{m}$
    \item $ac\equiv bd\mod{m}$
\end{itemize}

\textbf{Theorom 1:} For any $a,b \in \Z$ there exists $u,v \in \Z$ such that $au + bv = gcd(a,b)$. In words, we say that $a$ and $b$ can be written as a linear combination of their gcd.

\textbf{Theorem 2:} The congruence $ax \equiv c \pmod{m}$ has a solution if and only if the gcd$(a,m)$ divides $c$. 
\[
\gcd(a,m) \mid c.
\]

 
	\subsection*{Warm-Up}

	\begin{enumerate}[a.]
		\item Given $a \equiv b \pmod{m}$, prove $a + c\equiv b + c\pmod{m}$ for $c \in \mathbb{Z}$.

		\begin{mdframed}
		\vspace{2.8cm}
		\end{mdframed}


		\item Given $a \equiv b \pmod{m}$, prove $ac\equiv bc\pmod{m}$ for $c \in \mathbb{Z}$.


		\begin{mdframed}
		\vspace{4cm}
		\end{mdframed}

		\item Given $a \equiv b \pmod{m}$, prove $a^2\equiv b^2\pmod{m}$.

		\begin{mdframed}
		\vspace{5cm}
		\end{mdframed}


		\item For every odd integer $n$, prove that $n^4 -1$ is divisible by 8.
		
		\begin{mdframed}
		\vspace{5cm}
		\end{mdframed}
	\end{enumerate}

	\section*{Section Lesson - Modular Inverses and the Three's Trick}

	Say we are trying to solve for $x$ in the equation $8x = 2$, how would we do so?

	Answer: we would multiply both sides by $8^{-1} = \frac{1}{8}$. This is called the inverse of 8.
	
	\begin{align*}
		&\frac{1}{8}\cdot 8 \cdot x  = \frac{1}{8} \cdot 2 \\
		&\Rightarrow x = 0.25
	\end{align*}	

	And in general, if we are trying to solve for $x$ in the equaltion $ax = c$ we simply multiply both sides by $a^{-1} = \frac{1}{a}$. 

	The $a^{-1}$ notation indicates that $a^{-1}\cdot a = 1$. 

	However, it is not so simple when we are working with congruence relations.

	For example, there is \textbf{no solution} for $x$ in the equation $8x \equiv 2 \pmod{12}$. 

	In general, $ax \equiv c \pmod{m}$ has a solution if and only if $\gcd(a,m) \mid c$. (In english: if and only if the gcd of $a$ and $m$ divides $c$.)

	\begin{enumerate}[a.]
		\item Prove that if $gcd(a,m) \mid c$ then  $ax \equiv c \pmod{m}$ has a solution.
		
		\hint{Let $d = \gcd(a,m)$. From Theorom 1 above we know that $d = au + mv$ for some $u,v\in \Z$.}

		\hint{Since $d \mid c$ then $c = kd$ for some $k \in \Z$}

		\begin{mdframed}
			\vspace{10cm}
		\end{mdframed}

		\item Use the strategy you found above to solve for $4x \equiv 6 \pmod{14}$. 

Use the fact that $4\cdot 4 + (-1)\cdot 14 = 2$.

		\begin{mdframed}
			\vspace{6cm}
		\end{mdframed}
		
	\end{enumerate}

		\subsection*{Modular Inverses Explained}

		A modular inverse for $a$ mod $m$ is a number $a^{-1}$ such that $a \cdot a^{-1} \equiv 1 \pmod{m}$.

		In other words, a modular inverse for $a$ mod $m$ is the $x$ which solves $ax \equiv 1 \pmod{m}$.

		\begin{enumerate}[a.]
		\setcounter{enumi}{2}
			\item If $a$ has a modular inverse mod $m$ then what is $\gcd(a,m)$.
			\begin{mdframed}
			\vspace{1cm}
			\end{mdframed}

		\end{enumerate}

		A modular inverse is extremely helpful in solving equations $ax \equiv b \pmod{m}$.

		If $a$ has a modular inverse mod ${m}$ then $x \equiv a^{-1}b \pmod{m}$.

		
		\begin{enumerate}[a.]
					\setcounter{enumi}{3}
			\item Use the technique from question ``a" to find the modular inverse of $4$ mod 9.

			\hint Use the fact that 28-27 = 1.
			
			\begin{mdframed}
			\vspace{3cm}
			\end{mdframed}

			\item Use $4^{-1}$ to solve for $x$ in the equation $4x \equiv 3 \pmod{9}$. Verify your answer.

			\begin{mdframed}
			\vspace{3cm}
			\end{mdframed}
			
		\end{enumerate}

		\subsection*{The Threes Trick}

		Here is a trick to determine if a number $n$ is divisible by 3:
		
		\textit{``If the sum of the digits of $n$ is divisble by 3, so is $n$."}

		For example, 261 is divisble by 3 since $2+6+1 = 9$.

		You are going to prove this.

		\begin{enumerate}[a.]
							\setcounter{enumi}{5}
			\item For any $k \in \N$, what is $10^k$ congruent to mod 3?

			\begin{mdframed}
			\vspace{1cm}
			\end{mdframed}

			\item For any $k \in \N$, what is the modular inverse of $10^k$ mod 3>

			Recall the modular inverse is the $x$ which solves $10^k x\equiv 1 \pmod 3$

			\begin{mdframed}
			\vspace{1cm}
			\end{mdframed}

			\item Prove the ``Threes trick" by expanding a number in terms of its digits.


			\begin{mdframed}
			\vspace{5.5cm}
			\end{mdframed}



			\item Does the Threes trick work when we are not in Base 10? What numbers does it apply for in base $b$?
			

			\begin{mdframed}
			\vspace{5.5cm}
			\end{mdframed}
			
		\end{enumerate}

		

	
	

\end{document}